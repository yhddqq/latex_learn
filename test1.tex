% 12字体大小,a4纸,单面打印 ctexart 为article的派生,支持中文排版
% article为文章格式的文档;report为报告格式,用于综述,长篇论文;book为书籍
% proc为基于article一个简单学术文档模板,slides为幻灯格式文档,minimal为极简模式文档
\documentclass[12pt,a4paper,oneside]{ctexart}
% \documentclass[10pt,a4paper,twoide]{ctexart}默认10pt,letterpapaer

% 导言区
% 加载宏包,数学公式包
\usepackage{amsmath,amsthm,amssymb,graphicx}
% \usepackage{ctex}
% 加载超链接宏包hyperref,并进行相关设置
\usepackage[bookmarks=true,colorlinks,citecolor=blue,linkcolor=red]{hyperref}
% latex默认页边距如果很大,用geometry宏包可以显示更多内容
\usepackage{geometry}
\geometry{left=2.5cm,right=2.5cm,top=3cm,bottom=3.18cm}
% 加粗公式用bm宏包,这样可以保留公式的斜体
\usepackage{bm}

% 只导入某些文件
\includeonly{filelist}

% 设置行距
\linespread{1.5}
% 设置页码的数字字体:aiph为小写字母,Aiph为大写字母,Roman为大写罗马数字,arabic为默认阿拉伯数字
\pagenumbering{arabic}
% 设置页码从0开始
\setcounter{page}{0}
% 加入定理环境:{theorem}是环境的名称,{定理}设置了该环境显示的名称是“定理”,[section]的作用是让theorem环境在每个section中单独编号
\newtheorem{theorem}{定理}[section]
% 可以建立新的环境,如果要让新的环境和theorem环境一起计数
\newtheorem{definition}[theorem]{定义}
\newtheorem{lemma}[theorem]{引理}
\newtheorem{corollary}[theorem]{推论}
\newtheorem{example}[theorem]{例}
\newtheorem{proposition}[theorem]{命题}


\title{这是一个\LaTeX 模板}
\author{洛尘}
\date{\today}


% 正文区,真正的文章内容,所见所得部分
\begin{document}

% \begin{titlepage}
    
% \end{titlepage}

% 将导言区的标题,作者,日期等部分显示
\maketitle

% 在指定位置生成目录
\tableofcontents

\begin{abstract}
    摘要内容
\end{abstract}

正文内容:
第一段,hello,world
此处会和上面一行接上

第二段:注意,需要和上面空一行

\newpage
这是下一页的内容:hello,world,似乎latex还会自动加入页码

\textup{局部特殊字体:直立abcd}

\textit{局部特殊字体:意大利abcd}

\textsl{局部特殊字体:倾斜abcd}

\textsc{局部特殊字体:小型大写abcd}

% 对于ctexart类型的文章(第一行看的出来), 用\section{}以及\subsection{}标记章节
\section{一级标题}
nice to me too.
\subsection{二级标题}
我的纸飞机呀,飞呀飞
\subsubsection{三级标题}
hello,words

% 导入其他的,input是直接导入,include会换一页
% % 12字体大小,a4纸,单面打印 ctexart 为article的派生,支持中文排版
% article为文章格式的文档;report为报告格式,用于综述,长篇论文;book为书籍
% proc为基于article一个简单学术文档模板,slides为幻灯格式文档,minimal为极简模式文档
\documentclass[12pt,a4paper,oneside]{ctexart}
% \documentclass[10pt,a4paper,twoide]{ctexart}默认10pt,letterpapaer

% 导言区
% 加载宏包,数学公式包
\usepackage{amsmath,amsthm,amssymb,graphicx}
% \usepackage{ctex}
% 加载超链接宏包hyperref,并进行相关设置
\usepackage[bookmarks=true,colorlinks,citecolor=blue,linkcolor=red]{hyperref}
% latex默认页边距如果很大,用geometry宏包可以显示更多内容
\usepackage{geometry}
\geometry{left=2.5cm,right=2.5cm,top=3cm,bottom=3.18cm}
% 加粗公式用bm宏包,这样可以保留公式的斜体
\usepackage{bm}

% 只导入某些文件
\includeonly{filelist}

% 设置行距
\linespread{1.5}
% 设置页码的数字字体:aiph为小写字母,Aiph为大写字母,Roman为大写罗马数字,arabic为默认阿拉伯数字
\pagenumbering{arabic}
% 设置页码从0开始
\setcounter{page}{0}
% 加入定理环境:{theorem}是环境的名称,{定理}设置了该环境显示的名称是“定理”,[section]的作用是让theorem环境在每个section中单独编号
\newtheorem{theorem}{定理}[section]
% 可以建立新的环境,如果要让新的环境和theorem环境一起计数
\newtheorem{definition}[theorem]{定义}
\newtheorem{lemma}[theorem]{引理}
\newtheorem{corollary}[theorem]{推论}
\newtheorem{example}[theorem]{例}
\newtheorem{proposition}[theorem]{命题}


\title{这是一个\LaTeX 模板}
\author{洛尘}
\date{\today}


% 正文区,真正的文章内容,所见所得部分
\begin{document}

% \begin{titlepage}
    
% \end{titlepage}

% 将导言区的标题,作者,日期等部分显示
\maketitle

% 在指定位置生成目录
\tableofcontents

\begin{abstract}
    摘要内容
\end{abstract}

正文内容:
第一段,hello,world
此处会和上面一行接上

第二段:注意,需要和上面空一行

\newpage
这是下一页的内容:hello,world,似乎latex还会自动加入页码

\textup{局部特殊字体:直立abcd}

\textit{局部特殊字体:意大利abcd}

\textsl{局部特殊字体:倾斜abcd}

\textsc{局部特殊字体:小型大写abcd}

% 对于ctexart类型的文章(第一行看的出来), 用\section{}以及\subsection{}标记章节
\section{一级标题}
nice to me too.
\subsection{二级标题}
我的纸飞机呀,飞呀飞
\subsubsection{三级标题}
hello,words

% 导入其他的,input是直接导入,include会换一页
% % 12字体大小,a4纸,单面打印 ctexart 为article的派生,支持中文排版
% article为文章格式的文档;report为报告格式,用于综述,长篇论文;book为书籍
% proc为基于article一个简单学术文档模板,slides为幻灯格式文档,minimal为极简模式文档
\documentclass[12pt,a4paper,oneside]{ctexart}
% \documentclass[10pt,a4paper,twoide]{ctexart}默认10pt,letterpapaer

% 导言区
% 加载宏包,数学公式包
\usepackage{amsmath,amsthm,amssymb,graphicx}
% \usepackage{ctex}
% 加载超链接宏包hyperref,并进行相关设置
\usepackage[bookmarks=true,colorlinks,citecolor=blue,linkcolor=red]{hyperref}
% latex默认页边距如果很大,用geometry宏包可以显示更多内容
\usepackage{geometry}
\geometry{left=2.5cm,right=2.5cm,top=3cm,bottom=3.18cm}
% 加粗公式用bm宏包,这样可以保留公式的斜体
\usepackage{bm}

% 只导入某些文件
\includeonly{filelist}

% 设置行距
\linespread{1.5}
% 设置页码的数字字体:aiph为小写字母,Aiph为大写字母,Roman为大写罗马数字,arabic为默认阿拉伯数字
\pagenumbering{arabic}
% 设置页码从0开始
\setcounter{page}{0}
% 加入定理环境:{theorem}是环境的名称,{定理}设置了该环境显示的名称是“定理”,[section]的作用是让theorem环境在每个section中单独编号
\newtheorem{theorem}{定理}[section]
% 可以建立新的环境,如果要让新的环境和theorem环境一起计数
\newtheorem{definition}[theorem]{定义}
\newtheorem{lemma}[theorem]{引理}
\newtheorem{corollary}[theorem]{推论}
\newtheorem{example}[theorem]{例}
\newtheorem{proposition}[theorem]{命题}


\title{这是一个\LaTeX 模板}
\author{洛尘}
\date{\today}


% 正文区,真正的文章内容,所见所得部分
\begin{document}

% \begin{titlepage}
    
% \end{titlepage}

% 将导言区的标题,作者,日期等部分显示
\maketitle

% 在指定位置生成目录
\tableofcontents

\begin{abstract}
    摘要内容
\end{abstract}

正文内容:
第一段,hello,world
此处会和上面一行接上

第二段:注意,需要和上面空一行

\newpage
这是下一页的内容:hello,world,似乎latex还会自动加入页码

\textup{局部特殊字体:直立abcd}

\textit{局部特殊字体:意大利abcd}

\textsl{局部特殊字体:倾斜abcd}

\textsc{局部特殊字体:小型大写abcd}

% 对于ctexart类型的文章(第一行看的出来), 用\section{}以及\subsection{}标记章节
\section{一级标题}
nice to me too.
\subsection{二级标题}
我的纸飞机呀,飞呀飞
\subsubsection{三级标题}
hello,words

% 导入其他的,input是直接导入,include会换一页
% % 12字体大小,a4纸,单面打印 ctexart 为article的派生,支持中文排版
% article为文章格式的文档;report为报告格式,用于综述,长篇论文;book为书籍
% proc为基于article一个简单学术文档模板,slides为幻灯格式文档,minimal为极简模式文档
\documentclass[12pt,a4paper,oneside]{ctexart}
% \documentclass[10pt,a4paper,twoide]{ctexart}默认10pt,letterpapaer

% 导言区
% 加载宏包,数学公式包
\usepackage{amsmath,amsthm,amssymb,graphicx}
% \usepackage{ctex}
% 加载超链接宏包hyperref,并进行相关设置
\usepackage[bookmarks=true,colorlinks,citecolor=blue,linkcolor=red]{hyperref}
% latex默认页边距如果很大,用geometry宏包可以显示更多内容
\usepackage{geometry}
\geometry{left=2.5cm,right=2.5cm,top=3cm,bottom=3.18cm}
% 加粗公式用bm宏包,这样可以保留公式的斜体
\usepackage{bm}

% 只导入某些文件
\includeonly{filelist}

% 设置行距
\linespread{1.5}
% 设置页码的数字字体:aiph为小写字母,Aiph为大写字母,Roman为大写罗马数字,arabic为默认阿拉伯数字
\pagenumbering{arabic}
% 设置页码从0开始
\setcounter{page}{0}
% 加入定理环境:{theorem}是环境的名称,{定理}设置了该环境显示的名称是“定理”,[section]的作用是让theorem环境在每个section中单独编号
\newtheorem{theorem}{定理}[section]
% 可以建立新的环境,如果要让新的环境和theorem环境一起计数
\newtheorem{definition}[theorem]{定义}
\newtheorem{lemma}[theorem]{引理}
\newtheorem{corollary}[theorem]{推论}
\newtheorem{example}[theorem]{例}
\newtheorem{proposition}[theorem]{命题}


\title{这是一个\LaTeX 模板}
\author{洛尘}
\date{\today}


% 正文区,真正的文章内容,所见所得部分
\begin{document}

% \begin{titlepage}
    
% \end{titlepage}

% 将导言区的标题,作者,日期等部分显示
\maketitle

% 在指定位置生成目录
\tableofcontents

\begin{abstract}
    摘要内容
\end{abstract}

正文内容:
第一段,hello,world
此处会和上面一行接上

第二段:注意,需要和上面空一行

\newpage
这是下一页的内容:hello,world,似乎latex还会自动加入页码

\textup{局部特殊字体:直立abcd}

\textit{局部特殊字体:意大利abcd}

\textsl{局部特殊字体:倾斜abcd}

\textsc{局部特殊字体:小型大写abcd}

% 对于ctexart类型的文章(第一行看的出来), 用\section{}以及\subsection{}标记章节
\section{一级标题}
nice to me too.
\subsection{二级标题}
我的纸飞机呀,飞呀飞
\subsubsection{三级标题}
hello,words

% 导入其他的,input是直接导入,include会换一页
% \input{test1.tex}
% \include{test1}

\newpage
插入图片的使用方式:
\begin{figure}[htbp] % htbp表示自动选择插入图片的最优位置
    \centering % 设置让图片居中
    \includegraphics[width=8cm]{pics/雷电将军.jpg} % 这是图片宽度为8cm
    \caption{图片标题:雷神}
\end{figure}

\newpage
插入表格的使用方式:参考https://www.tablesgenerator.com/
\begin{table}[htbp]
    \centering 
    \caption{表格标题}
    \begin{tabular}{lllll}
    32  &     & \textbf{fsa} & sa  &     \\
    543 & fds & f            & d   & fds \\
    5   & fds & fds          &     & 4.2 \\
    4   &     & 6.065        & 2.5 &    
    \end{tabular}
\end{table}

加入列表,分为
1.无序列表
2.有序列表
3.描述description
\begin{enumerate}
    \item[(1)] 这是第一点;
    \item[(2)] 这是第二点,
    \item[(3)] 这是第三点。
\end{enumerate}

加入一条定理:
\begin{theorem}[定理名称:拉格朗日]
    定理内容:我想摆烂
\end{theorem}

试试建立新的环境
\begin{lemma}[theorem]{引理}
    这是一个引理
\end{lemma}
\begin{example}[theorem]{例}
    这是一个example
\end{example}
\begin{proposition}[theorem]{命题}
    这是一个命题
\end{proposition}

\section{数学公式}

数学公式:参考https://www.latexlive.com/

1.行内公式,用$..$输入
若$a>0$, $b>0$, 则$a+b>0$.

2.行间公式,用$$..$$输入
若$a>0$, $b>0$,则
$$
a+b>0.
$$

3.如果要在行内使用行间公式,则在前面加入displaystyle
设$\displaystyle\lim_{n\to\infty}x_n=x$

4.上下标用$x^7$, 下标用$y_7$,  内容多用$x^{abcd2}$, $y_{acsw5d}$.

5.分式: 5分之3, $\dfrac{3}{5}$, 
为了在分子分母行间输入较小的分式,可用$a^\frac{1}{n}$代替.

6.括号直接用$(...456+85)$输入, 如果括号内的内容高度很大,
比如出现分数的情况,可以改用$\left(1+a^\frac{1}{n}\right)^n$。
在中间需要隔开时,使用$\left(a^3\middle|x_6\right)$。
输入大括号时需要用\{大括号\}, 其中的反斜为转移字符。

7.加粗公式,注意不是文字,只针对公式: \bm{$3^4$}, \bm{$a^4$}, \textsl{\bm{$a^4$}}, \textsl{\bm{$a^x$}}

8.分段函数:
$$
f(x)=\begin{cases}
    x, & x>0, \\
    -x, & x\leq 0.
\end{cases}
$$

9.多行公式:
$$
\begin{aligned}
    a & =b+c \\
    & =d+e \\
    & = s+2
\end{aligned}
$$

10.行列式和矩阵 bmatrix和pmatrix
$$
\begin{bmatrix}
    a & b \\
    c & d  
\end{bmatrix}
$$
$$
\begin{pmatrix}
    1&2&4&8 \\
    3&a&9&a \\
    d&x^2&x&y_5
\end{pmatrix}
$$
$$
\begin{vmatrix}
    a&3 \\
    b&t^3
\end{vmatrix}
$$

\end{document}

% % 12字体大小,a4纸,单面打印 ctexart 为article的派生,支持中文排版
% article为文章格式的文档;report为报告格式,用于综述,长篇论文;book为书籍
% proc为基于article一个简单学术文档模板,slides为幻灯格式文档,minimal为极简模式文档
\documentclass[12pt,a4paper,oneside]{ctexart}
% \documentclass[10pt,a4paper,twoide]{ctexart}默认10pt,letterpapaer

% 导言区
% 加载宏包,数学公式包
\usepackage{amsmath,amsthm,amssymb,graphicx}
% \usepackage{ctex}
% 加载超链接宏包hyperref,并进行相关设置
\usepackage[bookmarks=true,colorlinks,citecolor=blue,linkcolor=red]{hyperref}
% latex默认页边距如果很大,用geometry宏包可以显示更多内容
\usepackage{geometry}
\geometry{left=2.5cm,right=2.5cm,top=3cm,bottom=3.18cm}
% 加粗公式用bm宏包,这样可以保留公式的斜体
\usepackage{bm}

% 只导入某些文件
\includeonly{filelist}

% 设置行距
\linespread{1.5}
% 设置页码的数字字体:aiph为小写字母,Aiph为大写字母,Roman为大写罗马数字,arabic为默认阿拉伯数字
\pagenumbering{arabic}
% 设置页码从0开始
\setcounter{page}{0}
% 加入定理环境:{theorem}是环境的名称,{定理}设置了该环境显示的名称是“定理”,[section]的作用是让theorem环境在每个section中单独编号
\newtheorem{theorem}{定理}[section]
% 可以建立新的环境,如果要让新的环境和theorem环境一起计数
\newtheorem{definition}[theorem]{定义}
\newtheorem{lemma}[theorem]{引理}
\newtheorem{corollary}[theorem]{推论}
\newtheorem{example}[theorem]{例}
\newtheorem{proposition}[theorem]{命题}


\title{这是一个\LaTeX 模板}
\author{洛尘}
\date{\today}


% 正文区,真正的文章内容,所见所得部分
\begin{document}

% \begin{titlepage}
    
% \end{titlepage}

% 将导言区的标题,作者,日期等部分显示
\maketitle

% 在指定位置生成目录
\tableofcontents

\begin{abstract}
    摘要内容
\end{abstract}

正文内容:
第一段,hello,world
此处会和上面一行接上

第二段:注意,需要和上面空一行

\newpage
这是下一页的内容:hello,world,似乎latex还会自动加入页码

\textup{局部特殊字体:直立abcd}

\textit{局部特殊字体:意大利abcd}

\textsl{局部特殊字体:倾斜abcd}

\textsc{局部特殊字体:小型大写abcd}

% 对于ctexart类型的文章(第一行看的出来), 用\section{}以及\subsection{}标记章节
\section{一级标题}
nice to me too.
\subsection{二级标题}
我的纸飞机呀,飞呀飞
\subsubsection{三级标题}
hello,words

% 导入其他的,input是直接导入,include会换一页
% \input{test1.tex}
% \include{test1}

\newpage
插入图片的使用方式:
\begin{figure}[htbp] % htbp表示自动选择插入图片的最优位置
    \centering % 设置让图片居中
    \includegraphics[width=8cm]{pics/雷电将军.jpg} % 这是图片宽度为8cm
    \caption{图片标题:雷神}
\end{figure}

\newpage
插入表格的使用方式:参考https://www.tablesgenerator.com/
\begin{table}[htbp]
    \centering 
    \caption{表格标题}
    \begin{tabular}{lllll}
    32  &     & \textbf{fsa} & sa  &     \\
    543 & fds & f            & d   & fds \\
    5   & fds & fds          &     & 4.2 \\
    4   &     & 6.065        & 2.5 &    
    \end{tabular}
\end{table}

加入列表,分为
1.无序列表
2.有序列表
3.描述description
\begin{enumerate}
    \item[(1)] 这是第一点;
    \item[(2)] 这是第二点,
    \item[(3)] 这是第三点。
\end{enumerate}

加入一条定理:
\begin{theorem}[定理名称:拉格朗日]
    定理内容:我想摆烂
\end{theorem}

试试建立新的环境
\begin{lemma}[theorem]{引理}
    这是一个引理
\end{lemma}
\begin{example}[theorem]{例}
    这是一个example
\end{example}
\begin{proposition}[theorem]{命题}
    这是一个命题
\end{proposition}

\section{数学公式}

数学公式:参考https://www.latexlive.com/

1.行内公式,用$..$输入
若$a>0$, $b>0$, 则$a+b>0$.

2.行间公式,用$$..$$输入
若$a>0$, $b>0$,则
$$
a+b>0.
$$

3.如果要在行内使用行间公式,则在前面加入displaystyle
设$\displaystyle\lim_{n\to\infty}x_n=x$

4.上下标用$x^7$, 下标用$y_7$,  内容多用$x^{abcd2}$, $y_{acsw5d}$.

5.分式: 5分之3, $\dfrac{3}{5}$, 
为了在分子分母行间输入较小的分式,可用$a^\frac{1}{n}$代替.

6.括号直接用$(...456+85)$输入, 如果括号内的内容高度很大,
比如出现分数的情况,可以改用$\left(1+a^\frac{1}{n}\right)^n$。
在中间需要隔开时,使用$\left(a^3\middle|x_6\right)$。
输入大括号时需要用\{大括号\}, 其中的反斜为转移字符。

7.加粗公式,注意不是文字,只针对公式: \bm{$3^4$}, \bm{$a^4$}, \textsl{\bm{$a^4$}}, \textsl{\bm{$a^x$}}

8.分段函数:
$$
f(x)=\begin{cases}
    x, & x>0, \\
    -x, & x\leq 0.
\end{cases}
$$

9.多行公式:
$$
\begin{aligned}
    a & =b+c \\
    & =d+e \\
    & = s+2
\end{aligned}
$$

10.行列式和矩阵 bmatrix和pmatrix
$$
\begin{bmatrix}
    a & b \\
    c & d  
\end{bmatrix}
$$
$$
\begin{pmatrix}
    1&2&4&8 \\
    3&a&9&a \\
    d&x^2&x&y_5
\end{pmatrix}
$$
$$
\begin{vmatrix}
    a&3 \\
    b&t^3
\end{vmatrix}
$$

\end{document}


\newpage
插入图片的使用方式:
\begin{figure}[htbp] % htbp表示自动选择插入图片的最优位置
    \centering % 设置让图片居中
    \includegraphics[width=8cm]{pics/雷电将军.jpg} % 这是图片宽度为8cm
    \caption{图片标题:雷神}
\end{figure}

\newpage
插入表格的使用方式:参考https://www.tablesgenerator.com/
\begin{table}[htbp]
    \centering 
    \caption{表格标题}
    \begin{tabular}{lllll}
    32  &     & \textbf{fsa} & sa  &     \\
    543 & fds & f            & d   & fds \\
    5   & fds & fds          &     & 4.2 \\
    4   &     & 6.065        & 2.5 &    
    \end{tabular}
\end{table}

加入列表,分为
1.无序列表
2.有序列表
3.描述description
\begin{enumerate}
    \item[(1)] 这是第一点;
    \item[(2)] 这是第二点,
    \item[(3)] 这是第三点。
\end{enumerate}

加入一条定理:
\begin{theorem}[定理名称:拉格朗日]
    定理内容:我想摆烂
\end{theorem}

试试建立新的环境
\begin{lemma}[theorem]{引理}
    这是一个引理
\end{lemma}
\begin{example}[theorem]{例}
    这是一个example
\end{example}
\begin{proposition}[theorem]{命题}
    这是一个命题
\end{proposition}

\section{数学公式}

数学公式:参考https://www.latexlive.com/

1.行内公式,用$..$输入
若$a>0$, $b>0$, 则$a+b>0$.

2.行间公式,用$$..$$输入
若$a>0$, $b>0$,则
$$
a+b>0.
$$

3.如果要在行内使用行间公式,则在前面加入displaystyle
设$\displaystyle\lim_{n\to\infty}x_n=x$

4.上下标用$x^7$, 下标用$y_7$,  内容多用$x^{abcd2}$, $y_{acsw5d}$.

5.分式: 5分之3, $\dfrac{3}{5}$, 
为了在分子分母行间输入较小的分式,可用$a^\frac{1}{n}$代替.

6.括号直接用$(...456+85)$输入, 如果括号内的内容高度很大,
比如出现分数的情况,可以改用$\left(1+a^\frac{1}{n}\right)^n$。
在中间需要隔开时,使用$\left(a^3\middle|x_6\right)$。
输入大括号时需要用\{大括号\}, 其中的反斜为转移字符。

7.加粗公式,注意不是文字,只针对公式: \bm{$3^4$}, \bm{$a^4$}, \textsl{\bm{$a^4$}}, \textsl{\bm{$a^x$}}

8.分段函数:
$$
f(x)=\begin{cases}
    x, & x>0, \\
    -x, & x\leq 0.
\end{cases}
$$

9.多行公式:
$$
\begin{aligned}
    a & =b+c \\
    & =d+e \\
    & = s+2
\end{aligned}
$$

10.行列式和矩阵 bmatrix和pmatrix
$$
\begin{bmatrix}
    a & b \\
    c & d  
\end{bmatrix}
$$
$$
\begin{pmatrix}
    1&2&4&8 \\
    3&a&9&a \\
    d&x^2&x&y_5
\end{pmatrix}
$$
$$
\begin{vmatrix}
    a&3 \\
    b&t^3
\end{vmatrix}
$$

\end{document}

% % 12字体大小,a4纸,单面打印 ctexart 为article的派生,支持中文排版
% article为文章格式的文档;report为报告格式,用于综述,长篇论文;book为书籍
% proc为基于article一个简单学术文档模板,slides为幻灯格式文档,minimal为极简模式文档
\documentclass[12pt,a4paper,oneside]{ctexart}
% \documentclass[10pt,a4paper,twoide]{ctexart}默认10pt,letterpapaer

% 导言区
% 加载宏包,数学公式包
\usepackage{amsmath,amsthm,amssymb,graphicx}
% \usepackage{ctex}
% 加载超链接宏包hyperref,并进行相关设置
\usepackage[bookmarks=true,colorlinks,citecolor=blue,linkcolor=red]{hyperref}
% latex默认页边距如果很大,用geometry宏包可以显示更多内容
\usepackage{geometry}
\geometry{left=2.5cm,right=2.5cm,top=3cm,bottom=3.18cm}
% 加粗公式用bm宏包,这样可以保留公式的斜体
\usepackage{bm}

% 只导入某些文件
\includeonly{filelist}

% 设置行距
\linespread{1.5}
% 设置页码的数字字体:aiph为小写字母,Aiph为大写字母,Roman为大写罗马数字,arabic为默认阿拉伯数字
\pagenumbering{arabic}
% 设置页码从0开始
\setcounter{page}{0}
% 加入定理环境:{theorem}是环境的名称,{定理}设置了该环境显示的名称是“定理”,[section]的作用是让theorem环境在每个section中单独编号
\newtheorem{theorem}{定理}[section]
% 可以建立新的环境,如果要让新的环境和theorem环境一起计数
\newtheorem{definition}[theorem]{定义}
\newtheorem{lemma}[theorem]{引理}
\newtheorem{corollary}[theorem]{推论}
\newtheorem{example}[theorem]{例}
\newtheorem{proposition}[theorem]{命题}


\title{这是一个\LaTeX 模板}
\author{洛尘}
\date{\today}


% 正文区,真正的文章内容,所见所得部分
\begin{document}

% \begin{titlepage}
    
% \end{titlepage}

% 将导言区的标题,作者,日期等部分显示
\maketitle

% 在指定位置生成目录
\tableofcontents

\begin{abstract}
    摘要内容
\end{abstract}

正文内容:
第一段,hello,world
此处会和上面一行接上

第二段:注意,需要和上面空一行

\newpage
这是下一页的内容:hello,world,似乎latex还会自动加入页码

\textup{局部特殊字体:直立abcd}

\textit{局部特殊字体:意大利abcd}

\textsl{局部特殊字体:倾斜abcd}

\textsc{局部特殊字体:小型大写abcd}

% 对于ctexart类型的文章(第一行看的出来), 用\section{}以及\subsection{}标记章节
\section{一级标题}
nice to me too.
\subsection{二级标题}
我的纸飞机呀,飞呀飞
\subsubsection{三级标题}
hello,words

% 导入其他的,input是直接导入,include会换一页
% % 12字体大小,a4纸,单面打印 ctexart 为article的派生,支持中文排版
% article为文章格式的文档;report为报告格式,用于综述,长篇论文;book为书籍
% proc为基于article一个简单学术文档模板,slides为幻灯格式文档,minimal为极简模式文档
\documentclass[12pt,a4paper,oneside]{ctexart}
% \documentclass[10pt,a4paper,twoide]{ctexart}默认10pt,letterpapaer

% 导言区
% 加载宏包,数学公式包
\usepackage{amsmath,amsthm,amssymb,graphicx}
% \usepackage{ctex}
% 加载超链接宏包hyperref,并进行相关设置
\usepackage[bookmarks=true,colorlinks,citecolor=blue,linkcolor=red]{hyperref}
% latex默认页边距如果很大,用geometry宏包可以显示更多内容
\usepackage{geometry}
\geometry{left=2.5cm,right=2.5cm,top=3cm,bottom=3.18cm}
% 加粗公式用bm宏包,这样可以保留公式的斜体
\usepackage{bm}

% 只导入某些文件
\includeonly{filelist}

% 设置行距
\linespread{1.5}
% 设置页码的数字字体:aiph为小写字母,Aiph为大写字母,Roman为大写罗马数字,arabic为默认阿拉伯数字
\pagenumbering{arabic}
% 设置页码从0开始
\setcounter{page}{0}
% 加入定理环境:{theorem}是环境的名称,{定理}设置了该环境显示的名称是“定理”,[section]的作用是让theorem环境在每个section中单独编号
\newtheorem{theorem}{定理}[section]
% 可以建立新的环境,如果要让新的环境和theorem环境一起计数
\newtheorem{definition}[theorem]{定义}
\newtheorem{lemma}[theorem]{引理}
\newtheorem{corollary}[theorem]{推论}
\newtheorem{example}[theorem]{例}
\newtheorem{proposition}[theorem]{命题}


\title{这是一个\LaTeX 模板}
\author{洛尘}
\date{\today}


% 正文区,真正的文章内容,所见所得部分
\begin{document}

% \begin{titlepage}
    
% \end{titlepage}

% 将导言区的标题,作者,日期等部分显示
\maketitle

% 在指定位置生成目录
\tableofcontents

\begin{abstract}
    摘要内容
\end{abstract}

正文内容:
第一段,hello,world
此处会和上面一行接上

第二段:注意,需要和上面空一行

\newpage
这是下一页的内容:hello,world,似乎latex还会自动加入页码

\textup{局部特殊字体:直立abcd}

\textit{局部特殊字体:意大利abcd}

\textsl{局部特殊字体:倾斜abcd}

\textsc{局部特殊字体:小型大写abcd}

% 对于ctexart类型的文章(第一行看的出来), 用\section{}以及\subsection{}标记章节
\section{一级标题}
nice to me too.
\subsection{二级标题}
我的纸飞机呀,飞呀飞
\subsubsection{三级标题}
hello,words

% 导入其他的,input是直接导入,include会换一页
% \input{test1.tex}
% \include{test1}

\newpage
插入图片的使用方式:
\begin{figure}[htbp] % htbp表示自动选择插入图片的最优位置
    \centering % 设置让图片居中
    \includegraphics[width=8cm]{pics/雷电将军.jpg} % 这是图片宽度为8cm
    \caption{图片标题:雷神}
\end{figure}

\newpage
插入表格的使用方式:参考https://www.tablesgenerator.com/
\begin{table}[htbp]
    \centering 
    \caption{表格标题}
    \begin{tabular}{lllll}
    32  &     & \textbf{fsa} & sa  &     \\
    543 & fds & f            & d   & fds \\
    5   & fds & fds          &     & 4.2 \\
    4   &     & 6.065        & 2.5 &    
    \end{tabular}
\end{table}

加入列表,分为
1.无序列表
2.有序列表
3.描述description
\begin{enumerate}
    \item[(1)] 这是第一点;
    \item[(2)] 这是第二点,
    \item[(3)] 这是第三点。
\end{enumerate}

加入一条定理:
\begin{theorem}[定理名称:拉格朗日]
    定理内容:我想摆烂
\end{theorem}

试试建立新的环境
\begin{lemma}[theorem]{引理}
    这是一个引理
\end{lemma}
\begin{example}[theorem]{例}
    这是一个example
\end{example}
\begin{proposition}[theorem]{命题}
    这是一个命题
\end{proposition}

\section{数学公式}

数学公式:参考https://www.latexlive.com/

1.行内公式,用$..$输入
若$a>0$, $b>0$, 则$a+b>0$.

2.行间公式,用$$..$$输入
若$a>0$, $b>0$,则
$$
a+b>0.
$$

3.如果要在行内使用行间公式,则在前面加入displaystyle
设$\displaystyle\lim_{n\to\infty}x_n=x$

4.上下标用$x^7$, 下标用$y_7$,  内容多用$x^{abcd2}$, $y_{acsw5d}$.

5.分式: 5分之3, $\dfrac{3}{5}$, 
为了在分子分母行间输入较小的分式,可用$a^\frac{1}{n}$代替.

6.括号直接用$(...456+85)$输入, 如果括号内的内容高度很大,
比如出现分数的情况,可以改用$\left(1+a^\frac{1}{n}\right)^n$。
在中间需要隔开时,使用$\left(a^3\middle|x_6\right)$。
输入大括号时需要用\{大括号\}, 其中的反斜为转移字符。

7.加粗公式,注意不是文字,只针对公式: \bm{$3^4$}, \bm{$a^4$}, \textsl{\bm{$a^4$}}, \textsl{\bm{$a^x$}}

8.分段函数:
$$
f(x)=\begin{cases}
    x, & x>0, \\
    -x, & x\leq 0.
\end{cases}
$$

9.多行公式:
$$
\begin{aligned}
    a & =b+c \\
    & =d+e \\
    & = s+2
\end{aligned}
$$

10.行列式和矩阵 bmatrix和pmatrix
$$
\begin{bmatrix}
    a & b \\
    c & d  
\end{bmatrix}
$$
$$
\begin{pmatrix}
    1&2&4&8 \\
    3&a&9&a \\
    d&x^2&x&y_5
\end{pmatrix}
$$
$$
\begin{vmatrix}
    a&3 \\
    b&t^3
\end{vmatrix}
$$

\end{document}

% % 12字体大小,a4纸,单面打印 ctexart 为article的派生,支持中文排版
% article为文章格式的文档;report为报告格式,用于综述,长篇论文;book为书籍
% proc为基于article一个简单学术文档模板,slides为幻灯格式文档,minimal为极简模式文档
\documentclass[12pt,a4paper,oneside]{ctexart}
% \documentclass[10pt,a4paper,twoide]{ctexart}默认10pt,letterpapaer

% 导言区
% 加载宏包,数学公式包
\usepackage{amsmath,amsthm,amssymb,graphicx}
% \usepackage{ctex}
% 加载超链接宏包hyperref,并进行相关设置
\usepackage[bookmarks=true,colorlinks,citecolor=blue,linkcolor=red]{hyperref}
% latex默认页边距如果很大,用geometry宏包可以显示更多内容
\usepackage{geometry}
\geometry{left=2.5cm,right=2.5cm,top=3cm,bottom=3.18cm}
% 加粗公式用bm宏包,这样可以保留公式的斜体
\usepackage{bm}

% 只导入某些文件
\includeonly{filelist}

% 设置行距
\linespread{1.5}
% 设置页码的数字字体:aiph为小写字母,Aiph为大写字母,Roman为大写罗马数字,arabic为默认阿拉伯数字
\pagenumbering{arabic}
% 设置页码从0开始
\setcounter{page}{0}
% 加入定理环境:{theorem}是环境的名称,{定理}设置了该环境显示的名称是“定理”,[section]的作用是让theorem环境在每个section中单独编号
\newtheorem{theorem}{定理}[section]
% 可以建立新的环境,如果要让新的环境和theorem环境一起计数
\newtheorem{definition}[theorem]{定义}
\newtheorem{lemma}[theorem]{引理}
\newtheorem{corollary}[theorem]{推论}
\newtheorem{example}[theorem]{例}
\newtheorem{proposition}[theorem]{命题}


\title{这是一个\LaTeX 模板}
\author{洛尘}
\date{\today}


% 正文区,真正的文章内容,所见所得部分
\begin{document}

% \begin{titlepage}
    
% \end{titlepage}

% 将导言区的标题,作者,日期等部分显示
\maketitle

% 在指定位置生成目录
\tableofcontents

\begin{abstract}
    摘要内容
\end{abstract}

正文内容:
第一段,hello,world
此处会和上面一行接上

第二段:注意,需要和上面空一行

\newpage
这是下一页的内容:hello,world,似乎latex还会自动加入页码

\textup{局部特殊字体:直立abcd}

\textit{局部特殊字体:意大利abcd}

\textsl{局部特殊字体:倾斜abcd}

\textsc{局部特殊字体:小型大写abcd}

% 对于ctexart类型的文章(第一行看的出来), 用\section{}以及\subsection{}标记章节
\section{一级标题}
nice to me too.
\subsection{二级标题}
我的纸飞机呀,飞呀飞
\subsubsection{三级标题}
hello,words

% 导入其他的,input是直接导入,include会换一页
% \input{test1.tex}
% \include{test1}

\newpage
插入图片的使用方式:
\begin{figure}[htbp] % htbp表示自动选择插入图片的最优位置
    \centering % 设置让图片居中
    \includegraphics[width=8cm]{pics/雷电将军.jpg} % 这是图片宽度为8cm
    \caption{图片标题:雷神}
\end{figure}

\newpage
插入表格的使用方式:参考https://www.tablesgenerator.com/
\begin{table}[htbp]
    \centering 
    \caption{表格标题}
    \begin{tabular}{lllll}
    32  &     & \textbf{fsa} & sa  &     \\
    543 & fds & f            & d   & fds \\
    5   & fds & fds          &     & 4.2 \\
    4   &     & 6.065        & 2.5 &    
    \end{tabular}
\end{table}

加入列表,分为
1.无序列表
2.有序列表
3.描述description
\begin{enumerate}
    \item[(1)] 这是第一点;
    \item[(2)] 这是第二点,
    \item[(3)] 这是第三点。
\end{enumerate}

加入一条定理:
\begin{theorem}[定理名称:拉格朗日]
    定理内容:我想摆烂
\end{theorem}

试试建立新的环境
\begin{lemma}[theorem]{引理}
    这是一个引理
\end{lemma}
\begin{example}[theorem]{例}
    这是一个example
\end{example}
\begin{proposition}[theorem]{命题}
    这是一个命题
\end{proposition}

\section{数学公式}

数学公式:参考https://www.latexlive.com/

1.行内公式,用$..$输入
若$a>0$, $b>0$, 则$a+b>0$.

2.行间公式,用$$..$$输入
若$a>0$, $b>0$,则
$$
a+b>0.
$$

3.如果要在行内使用行间公式,则在前面加入displaystyle
设$\displaystyle\lim_{n\to\infty}x_n=x$

4.上下标用$x^7$, 下标用$y_7$,  内容多用$x^{abcd2}$, $y_{acsw5d}$.

5.分式: 5分之3, $\dfrac{3}{5}$, 
为了在分子分母行间输入较小的分式,可用$a^\frac{1}{n}$代替.

6.括号直接用$(...456+85)$输入, 如果括号内的内容高度很大,
比如出现分数的情况,可以改用$\left(1+a^\frac{1}{n}\right)^n$。
在中间需要隔开时,使用$\left(a^3\middle|x_6\right)$。
输入大括号时需要用\{大括号\}, 其中的反斜为转移字符。

7.加粗公式,注意不是文字,只针对公式: \bm{$3^4$}, \bm{$a^4$}, \textsl{\bm{$a^4$}}, \textsl{\bm{$a^x$}}

8.分段函数:
$$
f(x)=\begin{cases}
    x, & x>0, \\
    -x, & x\leq 0.
\end{cases}
$$

9.多行公式:
$$
\begin{aligned}
    a & =b+c \\
    & =d+e \\
    & = s+2
\end{aligned}
$$

10.行列式和矩阵 bmatrix和pmatrix
$$
\begin{bmatrix}
    a & b \\
    c & d  
\end{bmatrix}
$$
$$
\begin{pmatrix}
    1&2&4&8 \\
    3&a&9&a \\
    d&x^2&x&y_5
\end{pmatrix}
$$
$$
\begin{vmatrix}
    a&3 \\
    b&t^3
\end{vmatrix}
$$

\end{document}


\newpage
插入图片的使用方式:
\begin{figure}[htbp] % htbp表示自动选择插入图片的最优位置
    \centering % 设置让图片居中
    \includegraphics[width=8cm]{pics/雷电将军.jpg} % 这是图片宽度为8cm
    \caption{图片标题:雷神}
\end{figure}

\newpage
插入表格的使用方式:参考https://www.tablesgenerator.com/
\begin{table}[htbp]
    \centering 
    \caption{表格标题}
    \begin{tabular}{lllll}
    32  &     & \textbf{fsa} & sa  &     \\
    543 & fds & f            & d   & fds \\
    5   & fds & fds          &     & 4.2 \\
    4   &     & 6.065        & 2.5 &    
    \end{tabular}
\end{table}

加入列表,分为
1.无序列表
2.有序列表
3.描述description
\begin{enumerate}
    \item[(1)] 这是第一点;
    \item[(2)] 这是第二点,
    \item[(3)] 这是第三点。
\end{enumerate}

加入一条定理:
\begin{theorem}[定理名称:拉格朗日]
    定理内容:我想摆烂
\end{theorem}

试试建立新的环境
\begin{lemma}[theorem]{引理}
    这是一个引理
\end{lemma}
\begin{example}[theorem]{例}
    这是一个example
\end{example}
\begin{proposition}[theorem]{命题}
    这是一个命题
\end{proposition}

\section{数学公式}

数学公式:参考https://www.latexlive.com/

1.行内公式,用$..$输入
若$a>0$, $b>0$, 则$a+b>0$.

2.行间公式,用$$..$$输入
若$a>0$, $b>0$,则
$$
a+b>0.
$$

3.如果要在行内使用行间公式,则在前面加入displaystyle
设$\displaystyle\lim_{n\to\infty}x_n=x$

4.上下标用$x^7$, 下标用$y_7$,  内容多用$x^{abcd2}$, $y_{acsw5d}$.

5.分式: 5分之3, $\dfrac{3}{5}$, 
为了在分子分母行间输入较小的分式,可用$a^\frac{1}{n}$代替.

6.括号直接用$(...456+85)$输入, 如果括号内的内容高度很大,
比如出现分数的情况,可以改用$\left(1+a^\frac{1}{n}\right)^n$。
在中间需要隔开时,使用$\left(a^3\middle|x_6\right)$。
输入大括号时需要用\{大括号\}, 其中的反斜为转移字符。

7.加粗公式,注意不是文字,只针对公式: \bm{$3^4$}, \bm{$a^4$}, \textsl{\bm{$a^4$}}, \textsl{\bm{$a^x$}}

8.分段函数:
$$
f(x)=\begin{cases}
    x, & x>0, \\
    -x, & x\leq 0.
\end{cases}
$$

9.多行公式:
$$
\begin{aligned}
    a & =b+c \\
    & =d+e \\
    & = s+2
\end{aligned}
$$

10.行列式和矩阵 bmatrix和pmatrix
$$
\begin{bmatrix}
    a & b \\
    c & d  
\end{bmatrix}
$$
$$
\begin{pmatrix}
    1&2&4&8 \\
    3&a&9&a \\
    d&x^2&x&y_5
\end{pmatrix}
$$
$$
\begin{vmatrix}
    a&3 \\
    b&t^3
\end{vmatrix}
$$

\end{document}


\newpage
插入图片的使用方式:
\begin{figure}[htbp] % htbp表示自动选择插入图片的最优位置
    \centering % 设置让图片居中
    \includegraphics[width=8cm]{pics/雷电将军.jpg} % 这是图片宽度为8cm
    \caption{图片标题:雷神}
\end{figure}

\newpage
插入表格的使用方式:参考https://www.tablesgenerator.com/
\begin{table}[htbp]
    \centering 
    \caption{表格标题}
    \begin{tabular}{lllll}
    32  &     & \textbf{fsa} & sa  &     \\
    543 & fds & f            & d   & fds \\
    5   & fds & fds          &     & 4.2 \\
    4   &     & 6.065        & 2.5 &    
    \end{tabular}
\end{table}

加入列表,分为
1.无序列表
2.有序列表
3.描述description
\begin{enumerate}
    \item[(1)] 这是第一点;
    \item[(2)] 这是第二点,
    \item[(3)] 这是第三点。
\end{enumerate}

加入一条定理:
\begin{theorem}[定理名称:拉格朗日]
    定理内容:我想摆烂
\end{theorem}

试试建立新的环境
\begin{lemma}[theorem]{引理}
    这是一个引理
\end{lemma}
\begin{example}[theorem]{例}
    这是一个example
\end{example}
\begin{proposition}[theorem]{命题}
    这是一个命题
\end{proposition}

\section{数学公式}

数学公式:参考https://www.latexlive.com/

1.行内公式,用$..$输入
若$a>0$, $b>0$, 则$a+b>0$.

2.行间公式,用$$..$$输入
若$a>0$, $b>0$,则
$$
a+b>0.
$$

3.如果要在行内使用行间公式,则在前面加入displaystyle
设$\displaystyle\lim_{n\to\infty}x_n=x$

4.上下标用$x^7$, 下标用$y_7$,  内容多用$x^{abcd2}$, $y_{acsw5d}$.

5.分式: 5分之3, $\dfrac{3}{5}$, 
为了在分子分母行间输入较小的分式,可用$a^\frac{1}{n}$代替.

6.括号直接用$(...456+85)$输入, 如果括号内的内容高度很大,
比如出现分数的情况,可以改用$\left(1+a^\frac{1}{n}\right)^n$。
在中间需要隔开时,使用$\left(a^3\middle|x_6\right)$。
输入大括号时需要用\{大括号\}, 其中的反斜为转移字符。

7.加粗公式,注意不是文字,只针对公式: \bm{$3^4$}, \bm{$a^4$}, \textsl{\bm{$a^4$}}, \textsl{\bm{$a^x$}}

8.分段函数:
$$
f(x)=\begin{cases}
    x, & x>0, \\
    -x, & x\leq 0.
\end{cases}
$$

9.多行公式:
$$
\begin{aligned}
    a & =b+c \\
    & =d+e \\
    & = s+2
\end{aligned}
$$

10.行列式和矩阵 bmatrix和pmatrix
$$
\begin{bmatrix}
    a & b \\
    c & d  
\end{bmatrix}
$$
$$
\begin{pmatrix}
    1&2&4&8 \\
    3&a&9&a \\
    d&x^2&x&y_5
\end{pmatrix}
$$
$$
\begin{vmatrix}
    a&3 \\
    b&t^3
\end{vmatrix}
$$

\end{document}

% % 12字体大小,a4纸,单面打印 ctexart 为article的派生,支持中文排版
% article为文章格式的文档;report为报告格式,用于综述,长篇论文;book为书籍
% proc为基于article一个简单学术文档模板,slides为幻灯格式文档,minimal为极简模式文档
\documentclass[12pt,a4paper,oneside]{ctexart}
% \documentclass[10pt,a4paper,twoide]{ctexart}默认10pt,letterpapaer

% 导言区
% 加载宏包,数学公式包
\usepackage{amsmath,amsthm,amssymb,graphicx}
% \usepackage{ctex}
% 加载超链接宏包hyperref,并进行相关设置
\usepackage[bookmarks=true,colorlinks,citecolor=blue,linkcolor=red]{hyperref}
% latex默认页边距如果很大,用geometry宏包可以显示更多内容
\usepackage{geometry}
\geometry{left=2.5cm,right=2.5cm,top=3cm,bottom=3.18cm}
% 加粗公式用bm宏包,这样可以保留公式的斜体
\usepackage{bm}

% 只导入某些文件
\includeonly{filelist}

% 设置行距
\linespread{1.5}
% 设置页码的数字字体:aiph为小写字母,Aiph为大写字母,Roman为大写罗马数字,arabic为默认阿拉伯数字
\pagenumbering{arabic}
% 设置页码从0开始
\setcounter{page}{0}
% 加入定理环境:{theorem}是环境的名称,{定理}设置了该环境显示的名称是“定理”,[section]的作用是让theorem环境在每个section中单独编号
\newtheorem{theorem}{定理}[section]
% 可以建立新的环境,如果要让新的环境和theorem环境一起计数
\newtheorem{definition}[theorem]{定义}
\newtheorem{lemma}[theorem]{引理}
\newtheorem{corollary}[theorem]{推论}
\newtheorem{example}[theorem]{例}
\newtheorem{proposition}[theorem]{命题}


\title{这是一个\LaTeX 模板}
\author{洛尘}
\date{\today}


% 正文区,真正的文章内容,所见所得部分
\begin{document}

% \begin{titlepage}
    
% \end{titlepage}

% 将导言区的标题,作者,日期等部分显示
\maketitle

% 在指定位置生成目录
\tableofcontents

\begin{abstract}
    摘要内容
\end{abstract}

正文内容:
第一段,hello,world
此处会和上面一行接上

第二段:注意,需要和上面空一行

\newpage
这是下一页的内容:hello,world,似乎latex还会自动加入页码

\textup{局部特殊字体:直立abcd}

\textit{局部特殊字体:意大利abcd}

\textsl{局部特殊字体:倾斜abcd}

\textsc{局部特殊字体:小型大写abcd}

% 对于ctexart类型的文章(第一行看的出来), 用\section{}以及\subsection{}标记章节
\section{一级标题}
nice to me too.
\subsection{二级标题}
我的纸飞机呀,飞呀飞
\subsubsection{三级标题}
hello,words

% 导入其他的,input是直接导入,include会换一页
% % 12字体大小,a4纸,单面打印 ctexart 为article的派生,支持中文排版
% article为文章格式的文档;report为报告格式,用于综述,长篇论文;book为书籍
% proc为基于article一个简单学术文档模板,slides为幻灯格式文档,minimal为极简模式文档
\documentclass[12pt,a4paper,oneside]{ctexart}
% \documentclass[10pt,a4paper,twoide]{ctexart}默认10pt,letterpapaer

% 导言区
% 加载宏包,数学公式包
\usepackage{amsmath,amsthm,amssymb,graphicx}
% \usepackage{ctex}
% 加载超链接宏包hyperref,并进行相关设置
\usepackage[bookmarks=true,colorlinks,citecolor=blue,linkcolor=red]{hyperref}
% latex默认页边距如果很大,用geometry宏包可以显示更多内容
\usepackage{geometry}
\geometry{left=2.5cm,right=2.5cm,top=3cm,bottom=3.18cm}
% 加粗公式用bm宏包,这样可以保留公式的斜体
\usepackage{bm}

% 只导入某些文件
\includeonly{filelist}

% 设置行距
\linespread{1.5}
% 设置页码的数字字体:aiph为小写字母,Aiph为大写字母,Roman为大写罗马数字,arabic为默认阿拉伯数字
\pagenumbering{arabic}
% 设置页码从0开始
\setcounter{page}{0}
% 加入定理环境:{theorem}是环境的名称,{定理}设置了该环境显示的名称是“定理”,[section]的作用是让theorem环境在每个section中单独编号
\newtheorem{theorem}{定理}[section]
% 可以建立新的环境,如果要让新的环境和theorem环境一起计数
\newtheorem{definition}[theorem]{定义}
\newtheorem{lemma}[theorem]{引理}
\newtheorem{corollary}[theorem]{推论}
\newtheorem{example}[theorem]{例}
\newtheorem{proposition}[theorem]{命题}


\title{这是一个\LaTeX 模板}
\author{洛尘}
\date{\today}


% 正文区,真正的文章内容,所见所得部分
\begin{document}

% \begin{titlepage}
    
% \end{titlepage}

% 将导言区的标题,作者,日期等部分显示
\maketitle

% 在指定位置生成目录
\tableofcontents

\begin{abstract}
    摘要内容
\end{abstract}

正文内容:
第一段,hello,world
此处会和上面一行接上

第二段:注意,需要和上面空一行

\newpage
这是下一页的内容:hello,world,似乎latex还会自动加入页码

\textup{局部特殊字体:直立abcd}

\textit{局部特殊字体:意大利abcd}

\textsl{局部特殊字体:倾斜abcd}

\textsc{局部特殊字体:小型大写abcd}

% 对于ctexart类型的文章(第一行看的出来), 用\section{}以及\subsection{}标记章节
\section{一级标题}
nice to me too.
\subsection{二级标题}
我的纸飞机呀,飞呀飞
\subsubsection{三级标题}
hello,words

% 导入其他的,input是直接导入,include会换一页
% % 12字体大小,a4纸,单面打印 ctexart 为article的派生,支持中文排版
% article为文章格式的文档;report为报告格式,用于综述,长篇论文;book为书籍
% proc为基于article一个简单学术文档模板,slides为幻灯格式文档,minimal为极简模式文档
\documentclass[12pt,a4paper,oneside]{ctexart}
% \documentclass[10pt,a4paper,twoide]{ctexart}默认10pt,letterpapaer

% 导言区
% 加载宏包,数学公式包
\usepackage{amsmath,amsthm,amssymb,graphicx}
% \usepackage{ctex}
% 加载超链接宏包hyperref,并进行相关设置
\usepackage[bookmarks=true,colorlinks,citecolor=blue,linkcolor=red]{hyperref}
% latex默认页边距如果很大,用geometry宏包可以显示更多内容
\usepackage{geometry}
\geometry{left=2.5cm,right=2.5cm,top=3cm,bottom=3.18cm}
% 加粗公式用bm宏包,这样可以保留公式的斜体
\usepackage{bm}

% 只导入某些文件
\includeonly{filelist}

% 设置行距
\linespread{1.5}
% 设置页码的数字字体:aiph为小写字母,Aiph为大写字母,Roman为大写罗马数字,arabic为默认阿拉伯数字
\pagenumbering{arabic}
% 设置页码从0开始
\setcounter{page}{0}
% 加入定理环境:{theorem}是环境的名称,{定理}设置了该环境显示的名称是“定理”,[section]的作用是让theorem环境在每个section中单独编号
\newtheorem{theorem}{定理}[section]
% 可以建立新的环境,如果要让新的环境和theorem环境一起计数
\newtheorem{definition}[theorem]{定义}
\newtheorem{lemma}[theorem]{引理}
\newtheorem{corollary}[theorem]{推论}
\newtheorem{example}[theorem]{例}
\newtheorem{proposition}[theorem]{命题}


\title{这是一个\LaTeX 模板}
\author{洛尘}
\date{\today}


% 正文区,真正的文章内容,所见所得部分
\begin{document}

% \begin{titlepage}
    
% \end{titlepage}

% 将导言区的标题,作者,日期等部分显示
\maketitle

% 在指定位置生成目录
\tableofcontents

\begin{abstract}
    摘要内容
\end{abstract}

正文内容:
第一段,hello,world
此处会和上面一行接上

第二段:注意,需要和上面空一行

\newpage
这是下一页的内容:hello,world,似乎latex还会自动加入页码

\textup{局部特殊字体:直立abcd}

\textit{局部特殊字体:意大利abcd}

\textsl{局部特殊字体:倾斜abcd}

\textsc{局部特殊字体:小型大写abcd}

% 对于ctexart类型的文章(第一行看的出来), 用\section{}以及\subsection{}标记章节
\section{一级标题}
nice to me too.
\subsection{二级标题}
我的纸飞机呀,飞呀飞
\subsubsection{三级标题}
hello,words

% 导入其他的,input是直接导入,include会换一页
% \input{test1.tex}
% \include{test1}

\newpage
插入图片的使用方式:
\begin{figure}[htbp] % htbp表示自动选择插入图片的最优位置
    \centering % 设置让图片居中
    \includegraphics[width=8cm]{pics/雷电将军.jpg} % 这是图片宽度为8cm
    \caption{图片标题:雷神}
\end{figure}

\newpage
插入表格的使用方式:参考https://www.tablesgenerator.com/
\begin{table}[htbp]
    \centering 
    \caption{表格标题}
    \begin{tabular}{lllll}
    32  &     & \textbf{fsa} & sa  &     \\
    543 & fds & f            & d   & fds \\
    5   & fds & fds          &     & 4.2 \\
    4   &     & 6.065        & 2.5 &    
    \end{tabular}
\end{table}

加入列表,分为
1.无序列表
2.有序列表
3.描述description
\begin{enumerate}
    \item[(1)] 这是第一点;
    \item[(2)] 这是第二点,
    \item[(3)] 这是第三点。
\end{enumerate}

加入一条定理:
\begin{theorem}[定理名称:拉格朗日]
    定理内容:我想摆烂
\end{theorem}

试试建立新的环境
\begin{lemma}[theorem]{引理}
    这是一个引理
\end{lemma}
\begin{example}[theorem]{例}
    这是一个example
\end{example}
\begin{proposition}[theorem]{命题}
    这是一个命题
\end{proposition}

\section{数学公式}

数学公式:参考https://www.latexlive.com/

1.行内公式,用$..$输入
若$a>0$, $b>0$, 则$a+b>0$.

2.行间公式,用$$..$$输入
若$a>0$, $b>0$,则
$$
a+b>0.
$$

3.如果要在行内使用行间公式,则在前面加入displaystyle
设$\displaystyle\lim_{n\to\infty}x_n=x$

4.上下标用$x^7$, 下标用$y_7$,  内容多用$x^{abcd2}$, $y_{acsw5d}$.

5.分式: 5分之3, $\dfrac{3}{5}$, 
为了在分子分母行间输入较小的分式,可用$a^\frac{1}{n}$代替.

6.括号直接用$(...456+85)$输入, 如果括号内的内容高度很大,
比如出现分数的情况,可以改用$\left(1+a^\frac{1}{n}\right)^n$。
在中间需要隔开时,使用$\left(a^3\middle|x_6\right)$。
输入大括号时需要用\{大括号\}, 其中的反斜为转移字符。

7.加粗公式,注意不是文字,只针对公式: \bm{$3^4$}, \bm{$a^4$}, \textsl{\bm{$a^4$}}, \textsl{\bm{$a^x$}}

8.分段函数:
$$
f(x)=\begin{cases}
    x, & x>0, \\
    -x, & x\leq 0.
\end{cases}
$$

9.多行公式:
$$
\begin{aligned}
    a & =b+c \\
    & =d+e \\
    & = s+2
\end{aligned}
$$

10.行列式和矩阵 bmatrix和pmatrix
$$
\begin{bmatrix}
    a & b \\
    c & d  
\end{bmatrix}
$$
$$
\begin{pmatrix}
    1&2&4&8 \\
    3&a&9&a \\
    d&x^2&x&y_5
\end{pmatrix}
$$
$$
\begin{vmatrix}
    a&3 \\
    b&t^3
\end{vmatrix}
$$

\end{document}

% % 12字体大小,a4纸,单面打印 ctexart 为article的派生,支持中文排版
% article为文章格式的文档;report为报告格式,用于综述,长篇论文;book为书籍
% proc为基于article一个简单学术文档模板,slides为幻灯格式文档,minimal为极简模式文档
\documentclass[12pt,a4paper,oneside]{ctexart}
% \documentclass[10pt,a4paper,twoide]{ctexart}默认10pt,letterpapaer

% 导言区
% 加载宏包,数学公式包
\usepackage{amsmath,amsthm,amssymb,graphicx}
% \usepackage{ctex}
% 加载超链接宏包hyperref,并进行相关设置
\usepackage[bookmarks=true,colorlinks,citecolor=blue,linkcolor=red]{hyperref}
% latex默认页边距如果很大,用geometry宏包可以显示更多内容
\usepackage{geometry}
\geometry{left=2.5cm,right=2.5cm,top=3cm,bottom=3.18cm}
% 加粗公式用bm宏包,这样可以保留公式的斜体
\usepackage{bm}

% 只导入某些文件
\includeonly{filelist}

% 设置行距
\linespread{1.5}
% 设置页码的数字字体:aiph为小写字母,Aiph为大写字母,Roman为大写罗马数字,arabic为默认阿拉伯数字
\pagenumbering{arabic}
% 设置页码从0开始
\setcounter{page}{0}
% 加入定理环境:{theorem}是环境的名称,{定理}设置了该环境显示的名称是“定理”,[section]的作用是让theorem环境在每个section中单独编号
\newtheorem{theorem}{定理}[section]
% 可以建立新的环境,如果要让新的环境和theorem环境一起计数
\newtheorem{definition}[theorem]{定义}
\newtheorem{lemma}[theorem]{引理}
\newtheorem{corollary}[theorem]{推论}
\newtheorem{example}[theorem]{例}
\newtheorem{proposition}[theorem]{命题}


\title{这是一个\LaTeX 模板}
\author{洛尘}
\date{\today}


% 正文区,真正的文章内容,所见所得部分
\begin{document}

% \begin{titlepage}
    
% \end{titlepage}

% 将导言区的标题,作者,日期等部分显示
\maketitle

% 在指定位置生成目录
\tableofcontents

\begin{abstract}
    摘要内容
\end{abstract}

正文内容:
第一段,hello,world
此处会和上面一行接上

第二段:注意,需要和上面空一行

\newpage
这是下一页的内容:hello,world,似乎latex还会自动加入页码

\textup{局部特殊字体:直立abcd}

\textit{局部特殊字体:意大利abcd}

\textsl{局部特殊字体:倾斜abcd}

\textsc{局部特殊字体:小型大写abcd}

% 对于ctexart类型的文章(第一行看的出来), 用\section{}以及\subsection{}标记章节
\section{一级标题}
nice to me too.
\subsection{二级标题}
我的纸飞机呀,飞呀飞
\subsubsection{三级标题}
hello,words

% 导入其他的,input是直接导入,include会换一页
% \input{test1.tex}
% \include{test1}

\newpage
插入图片的使用方式:
\begin{figure}[htbp] % htbp表示自动选择插入图片的最优位置
    \centering % 设置让图片居中
    \includegraphics[width=8cm]{pics/雷电将军.jpg} % 这是图片宽度为8cm
    \caption{图片标题:雷神}
\end{figure}

\newpage
插入表格的使用方式:参考https://www.tablesgenerator.com/
\begin{table}[htbp]
    \centering 
    \caption{表格标题}
    \begin{tabular}{lllll}
    32  &     & \textbf{fsa} & sa  &     \\
    543 & fds & f            & d   & fds \\
    5   & fds & fds          &     & 4.2 \\
    4   &     & 6.065        & 2.5 &    
    \end{tabular}
\end{table}

加入列表,分为
1.无序列表
2.有序列表
3.描述description
\begin{enumerate}
    \item[(1)] 这是第一点;
    \item[(2)] 这是第二点,
    \item[(3)] 这是第三点。
\end{enumerate}

加入一条定理:
\begin{theorem}[定理名称:拉格朗日]
    定理内容:我想摆烂
\end{theorem}

试试建立新的环境
\begin{lemma}[theorem]{引理}
    这是一个引理
\end{lemma}
\begin{example}[theorem]{例}
    这是一个example
\end{example}
\begin{proposition}[theorem]{命题}
    这是一个命题
\end{proposition}

\section{数学公式}

数学公式:参考https://www.latexlive.com/

1.行内公式,用$..$输入
若$a>0$, $b>0$, 则$a+b>0$.

2.行间公式,用$$..$$输入
若$a>0$, $b>0$,则
$$
a+b>0.
$$

3.如果要在行内使用行间公式,则在前面加入displaystyle
设$\displaystyle\lim_{n\to\infty}x_n=x$

4.上下标用$x^7$, 下标用$y_7$,  内容多用$x^{abcd2}$, $y_{acsw5d}$.

5.分式: 5分之3, $\dfrac{3}{5}$, 
为了在分子分母行间输入较小的分式,可用$a^\frac{1}{n}$代替.

6.括号直接用$(...456+85)$输入, 如果括号内的内容高度很大,
比如出现分数的情况,可以改用$\left(1+a^\frac{1}{n}\right)^n$。
在中间需要隔开时,使用$\left(a^3\middle|x_6\right)$。
输入大括号时需要用\{大括号\}, 其中的反斜为转移字符。

7.加粗公式,注意不是文字,只针对公式: \bm{$3^4$}, \bm{$a^4$}, \textsl{\bm{$a^4$}}, \textsl{\bm{$a^x$}}

8.分段函数:
$$
f(x)=\begin{cases}
    x, & x>0, \\
    -x, & x\leq 0.
\end{cases}
$$

9.多行公式:
$$
\begin{aligned}
    a & =b+c \\
    & =d+e \\
    & = s+2
\end{aligned}
$$

10.行列式和矩阵 bmatrix和pmatrix
$$
\begin{bmatrix}
    a & b \\
    c & d  
\end{bmatrix}
$$
$$
\begin{pmatrix}
    1&2&4&8 \\
    3&a&9&a \\
    d&x^2&x&y_5
\end{pmatrix}
$$
$$
\begin{vmatrix}
    a&3 \\
    b&t^3
\end{vmatrix}
$$

\end{document}


\newpage
插入图片的使用方式:
\begin{figure}[htbp] % htbp表示自动选择插入图片的最优位置
    \centering % 设置让图片居中
    \includegraphics[width=8cm]{pics/雷电将军.jpg} % 这是图片宽度为8cm
    \caption{图片标题:雷神}
\end{figure}

\newpage
插入表格的使用方式:参考https://www.tablesgenerator.com/
\begin{table}[htbp]
    \centering 
    \caption{表格标题}
    \begin{tabular}{lllll}
    32  &     & \textbf{fsa} & sa  &     \\
    543 & fds & f            & d   & fds \\
    5   & fds & fds          &     & 4.2 \\
    4   &     & 6.065        & 2.5 &    
    \end{tabular}
\end{table}

加入列表,分为
1.无序列表
2.有序列表
3.描述description
\begin{enumerate}
    \item[(1)] 这是第一点;
    \item[(2)] 这是第二点,
    \item[(3)] 这是第三点。
\end{enumerate}

加入一条定理:
\begin{theorem}[定理名称:拉格朗日]
    定理内容:我想摆烂
\end{theorem}

试试建立新的环境
\begin{lemma}[theorem]{引理}
    这是一个引理
\end{lemma}
\begin{example}[theorem]{例}
    这是一个example
\end{example}
\begin{proposition}[theorem]{命题}
    这是一个命题
\end{proposition}

\section{数学公式}

数学公式:参考https://www.latexlive.com/

1.行内公式,用$..$输入
若$a>0$, $b>0$, 则$a+b>0$.

2.行间公式,用$$..$$输入
若$a>0$, $b>0$,则
$$
a+b>0.
$$

3.如果要在行内使用行间公式,则在前面加入displaystyle
设$\displaystyle\lim_{n\to\infty}x_n=x$

4.上下标用$x^7$, 下标用$y_7$,  内容多用$x^{abcd2}$, $y_{acsw5d}$.

5.分式: 5分之3, $\dfrac{3}{5}$, 
为了在分子分母行间输入较小的分式,可用$a^\frac{1}{n}$代替.

6.括号直接用$(...456+85)$输入, 如果括号内的内容高度很大,
比如出现分数的情况,可以改用$\left(1+a^\frac{1}{n}\right)^n$。
在中间需要隔开时,使用$\left(a^3\middle|x_6\right)$。
输入大括号时需要用\{大括号\}, 其中的反斜为转移字符。

7.加粗公式,注意不是文字,只针对公式: \bm{$3^4$}, \bm{$a^4$}, \textsl{\bm{$a^4$}}, \textsl{\bm{$a^x$}}

8.分段函数:
$$
f(x)=\begin{cases}
    x, & x>0, \\
    -x, & x\leq 0.
\end{cases}
$$

9.多行公式:
$$
\begin{aligned}
    a & =b+c \\
    & =d+e \\
    & = s+2
\end{aligned}
$$

10.行列式和矩阵 bmatrix和pmatrix
$$
\begin{bmatrix}
    a & b \\
    c & d  
\end{bmatrix}
$$
$$
\begin{pmatrix}
    1&2&4&8 \\
    3&a&9&a \\
    d&x^2&x&y_5
\end{pmatrix}
$$
$$
\begin{vmatrix}
    a&3 \\
    b&t^3
\end{vmatrix}
$$

\end{document}

% % 12字体大小,a4纸,单面打印 ctexart 为article的派生,支持中文排版
% article为文章格式的文档;report为报告格式,用于综述,长篇论文;book为书籍
% proc为基于article一个简单学术文档模板,slides为幻灯格式文档,minimal为极简模式文档
\documentclass[12pt,a4paper,oneside]{ctexart}
% \documentclass[10pt,a4paper,twoide]{ctexart}默认10pt,letterpapaer

% 导言区
% 加载宏包,数学公式包
\usepackage{amsmath,amsthm,amssymb,graphicx}
% \usepackage{ctex}
% 加载超链接宏包hyperref,并进行相关设置
\usepackage[bookmarks=true,colorlinks,citecolor=blue,linkcolor=red]{hyperref}
% latex默认页边距如果很大,用geometry宏包可以显示更多内容
\usepackage{geometry}
\geometry{left=2.5cm,right=2.5cm,top=3cm,bottom=3.18cm}
% 加粗公式用bm宏包,这样可以保留公式的斜体
\usepackage{bm}

% 只导入某些文件
\includeonly{filelist}

% 设置行距
\linespread{1.5}
% 设置页码的数字字体:aiph为小写字母,Aiph为大写字母,Roman为大写罗马数字,arabic为默认阿拉伯数字
\pagenumbering{arabic}
% 设置页码从0开始
\setcounter{page}{0}
% 加入定理环境:{theorem}是环境的名称,{定理}设置了该环境显示的名称是“定理”,[section]的作用是让theorem环境在每个section中单独编号
\newtheorem{theorem}{定理}[section]
% 可以建立新的环境,如果要让新的环境和theorem环境一起计数
\newtheorem{definition}[theorem]{定义}
\newtheorem{lemma}[theorem]{引理}
\newtheorem{corollary}[theorem]{推论}
\newtheorem{example}[theorem]{例}
\newtheorem{proposition}[theorem]{命题}


\title{这是一个\LaTeX 模板}
\author{洛尘}
\date{\today}


% 正文区,真正的文章内容,所见所得部分
\begin{document}

% \begin{titlepage}
    
% \end{titlepage}

% 将导言区的标题,作者,日期等部分显示
\maketitle

% 在指定位置生成目录
\tableofcontents

\begin{abstract}
    摘要内容
\end{abstract}

正文内容:
第一段,hello,world
此处会和上面一行接上

第二段:注意,需要和上面空一行

\newpage
这是下一页的内容:hello,world,似乎latex还会自动加入页码

\textup{局部特殊字体:直立abcd}

\textit{局部特殊字体:意大利abcd}

\textsl{局部特殊字体:倾斜abcd}

\textsc{局部特殊字体:小型大写abcd}

% 对于ctexart类型的文章(第一行看的出来), 用\section{}以及\subsection{}标记章节
\section{一级标题}
nice to me too.
\subsection{二级标题}
我的纸飞机呀,飞呀飞
\subsubsection{三级标题}
hello,words

% 导入其他的,input是直接导入,include会换一页
% % 12字体大小,a4纸,单面打印 ctexart 为article的派生,支持中文排版
% article为文章格式的文档;report为报告格式,用于综述,长篇论文;book为书籍
% proc为基于article一个简单学术文档模板,slides为幻灯格式文档,minimal为极简模式文档
\documentclass[12pt,a4paper,oneside]{ctexart}
% \documentclass[10pt,a4paper,twoide]{ctexart}默认10pt,letterpapaer

% 导言区
% 加载宏包,数学公式包
\usepackage{amsmath,amsthm,amssymb,graphicx}
% \usepackage{ctex}
% 加载超链接宏包hyperref,并进行相关设置
\usepackage[bookmarks=true,colorlinks,citecolor=blue,linkcolor=red]{hyperref}
% latex默认页边距如果很大,用geometry宏包可以显示更多内容
\usepackage{geometry}
\geometry{left=2.5cm,right=2.5cm,top=3cm,bottom=3.18cm}
% 加粗公式用bm宏包,这样可以保留公式的斜体
\usepackage{bm}

% 只导入某些文件
\includeonly{filelist}

% 设置行距
\linespread{1.5}
% 设置页码的数字字体:aiph为小写字母,Aiph为大写字母,Roman为大写罗马数字,arabic为默认阿拉伯数字
\pagenumbering{arabic}
% 设置页码从0开始
\setcounter{page}{0}
% 加入定理环境:{theorem}是环境的名称,{定理}设置了该环境显示的名称是“定理”,[section]的作用是让theorem环境在每个section中单独编号
\newtheorem{theorem}{定理}[section]
% 可以建立新的环境,如果要让新的环境和theorem环境一起计数
\newtheorem{definition}[theorem]{定义}
\newtheorem{lemma}[theorem]{引理}
\newtheorem{corollary}[theorem]{推论}
\newtheorem{example}[theorem]{例}
\newtheorem{proposition}[theorem]{命题}


\title{这是一个\LaTeX 模板}
\author{洛尘}
\date{\today}


% 正文区,真正的文章内容,所见所得部分
\begin{document}

% \begin{titlepage}
    
% \end{titlepage}

% 将导言区的标题,作者,日期等部分显示
\maketitle

% 在指定位置生成目录
\tableofcontents

\begin{abstract}
    摘要内容
\end{abstract}

正文内容:
第一段,hello,world
此处会和上面一行接上

第二段:注意,需要和上面空一行

\newpage
这是下一页的内容:hello,world,似乎latex还会自动加入页码

\textup{局部特殊字体:直立abcd}

\textit{局部特殊字体:意大利abcd}

\textsl{局部特殊字体:倾斜abcd}

\textsc{局部特殊字体:小型大写abcd}

% 对于ctexart类型的文章(第一行看的出来), 用\section{}以及\subsection{}标记章节
\section{一级标题}
nice to me too.
\subsection{二级标题}
我的纸飞机呀,飞呀飞
\subsubsection{三级标题}
hello,words

% 导入其他的,input是直接导入,include会换一页
% \input{test1.tex}
% \include{test1}

\newpage
插入图片的使用方式:
\begin{figure}[htbp] % htbp表示自动选择插入图片的最优位置
    \centering % 设置让图片居中
    \includegraphics[width=8cm]{pics/雷电将军.jpg} % 这是图片宽度为8cm
    \caption{图片标题:雷神}
\end{figure}

\newpage
插入表格的使用方式:参考https://www.tablesgenerator.com/
\begin{table}[htbp]
    \centering 
    \caption{表格标题}
    \begin{tabular}{lllll}
    32  &     & \textbf{fsa} & sa  &     \\
    543 & fds & f            & d   & fds \\
    5   & fds & fds          &     & 4.2 \\
    4   &     & 6.065        & 2.5 &    
    \end{tabular}
\end{table}

加入列表,分为
1.无序列表
2.有序列表
3.描述description
\begin{enumerate}
    \item[(1)] 这是第一点;
    \item[(2)] 这是第二点,
    \item[(3)] 这是第三点。
\end{enumerate}

加入一条定理:
\begin{theorem}[定理名称:拉格朗日]
    定理内容:我想摆烂
\end{theorem}

试试建立新的环境
\begin{lemma}[theorem]{引理}
    这是一个引理
\end{lemma}
\begin{example}[theorem]{例}
    这是一个example
\end{example}
\begin{proposition}[theorem]{命题}
    这是一个命题
\end{proposition}

\section{数学公式}

数学公式:参考https://www.latexlive.com/

1.行内公式,用$..$输入
若$a>0$, $b>0$, 则$a+b>0$.

2.行间公式,用$$..$$输入
若$a>0$, $b>0$,则
$$
a+b>0.
$$

3.如果要在行内使用行间公式,则在前面加入displaystyle
设$\displaystyle\lim_{n\to\infty}x_n=x$

4.上下标用$x^7$, 下标用$y_7$,  内容多用$x^{abcd2}$, $y_{acsw5d}$.

5.分式: 5分之3, $\dfrac{3}{5}$, 
为了在分子分母行间输入较小的分式,可用$a^\frac{1}{n}$代替.

6.括号直接用$(...456+85)$输入, 如果括号内的内容高度很大,
比如出现分数的情况,可以改用$\left(1+a^\frac{1}{n}\right)^n$。
在中间需要隔开时,使用$\left(a^3\middle|x_6\right)$。
输入大括号时需要用\{大括号\}, 其中的反斜为转移字符。

7.加粗公式,注意不是文字,只针对公式: \bm{$3^4$}, \bm{$a^4$}, \textsl{\bm{$a^4$}}, \textsl{\bm{$a^x$}}

8.分段函数:
$$
f(x)=\begin{cases}
    x, & x>0, \\
    -x, & x\leq 0.
\end{cases}
$$

9.多行公式:
$$
\begin{aligned}
    a & =b+c \\
    & =d+e \\
    & = s+2
\end{aligned}
$$

10.行列式和矩阵 bmatrix和pmatrix
$$
\begin{bmatrix}
    a & b \\
    c & d  
\end{bmatrix}
$$
$$
\begin{pmatrix}
    1&2&4&8 \\
    3&a&9&a \\
    d&x^2&x&y_5
\end{pmatrix}
$$
$$
\begin{vmatrix}
    a&3 \\
    b&t^3
\end{vmatrix}
$$

\end{document}

% % 12字体大小,a4纸,单面打印 ctexart 为article的派生,支持中文排版
% article为文章格式的文档;report为报告格式,用于综述,长篇论文;book为书籍
% proc为基于article一个简单学术文档模板,slides为幻灯格式文档,minimal为极简模式文档
\documentclass[12pt,a4paper,oneside]{ctexart}
% \documentclass[10pt,a4paper,twoide]{ctexart}默认10pt,letterpapaer

% 导言区
% 加载宏包,数学公式包
\usepackage{amsmath,amsthm,amssymb,graphicx}
% \usepackage{ctex}
% 加载超链接宏包hyperref,并进行相关设置
\usepackage[bookmarks=true,colorlinks,citecolor=blue,linkcolor=red]{hyperref}
% latex默认页边距如果很大,用geometry宏包可以显示更多内容
\usepackage{geometry}
\geometry{left=2.5cm,right=2.5cm,top=3cm,bottom=3.18cm}
% 加粗公式用bm宏包,这样可以保留公式的斜体
\usepackage{bm}

% 只导入某些文件
\includeonly{filelist}

% 设置行距
\linespread{1.5}
% 设置页码的数字字体:aiph为小写字母,Aiph为大写字母,Roman为大写罗马数字,arabic为默认阿拉伯数字
\pagenumbering{arabic}
% 设置页码从0开始
\setcounter{page}{0}
% 加入定理环境:{theorem}是环境的名称,{定理}设置了该环境显示的名称是“定理”,[section]的作用是让theorem环境在每个section中单独编号
\newtheorem{theorem}{定理}[section]
% 可以建立新的环境,如果要让新的环境和theorem环境一起计数
\newtheorem{definition}[theorem]{定义}
\newtheorem{lemma}[theorem]{引理}
\newtheorem{corollary}[theorem]{推论}
\newtheorem{example}[theorem]{例}
\newtheorem{proposition}[theorem]{命题}


\title{这是一个\LaTeX 模板}
\author{洛尘}
\date{\today}


% 正文区,真正的文章内容,所见所得部分
\begin{document}

% \begin{titlepage}
    
% \end{titlepage}

% 将导言区的标题,作者,日期等部分显示
\maketitle

% 在指定位置生成目录
\tableofcontents

\begin{abstract}
    摘要内容
\end{abstract}

正文内容:
第一段,hello,world
此处会和上面一行接上

第二段:注意,需要和上面空一行

\newpage
这是下一页的内容:hello,world,似乎latex还会自动加入页码

\textup{局部特殊字体:直立abcd}

\textit{局部特殊字体:意大利abcd}

\textsl{局部特殊字体:倾斜abcd}

\textsc{局部特殊字体:小型大写abcd}

% 对于ctexart类型的文章(第一行看的出来), 用\section{}以及\subsection{}标记章节
\section{一级标题}
nice to me too.
\subsection{二级标题}
我的纸飞机呀,飞呀飞
\subsubsection{三级标题}
hello,words

% 导入其他的,input是直接导入,include会换一页
% \input{test1.tex}
% \include{test1}

\newpage
插入图片的使用方式:
\begin{figure}[htbp] % htbp表示自动选择插入图片的最优位置
    \centering % 设置让图片居中
    \includegraphics[width=8cm]{pics/雷电将军.jpg} % 这是图片宽度为8cm
    \caption{图片标题:雷神}
\end{figure}

\newpage
插入表格的使用方式:参考https://www.tablesgenerator.com/
\begin{table}[htbp]
    \centering 
    \caption{表格标题}
    \begin{tabular}{lllll}
    32  &     & \textbf{fsa} & sa  &     \\
    543 & fds & f            & d   & fds \\
    5   & fds & fds          &     & 4.2 \\
    4   &     & 6.065        & 2.5 &    
    \end{tabular}
\end{table}

加入列表,分为
1.无序列表
2.有序列表
3.描述description
\begin{enumerate}
    \item[(1)] 这是第一点;
    \item[(2)] 这是第二点,
    \item[(3)] 这是第三点。
\end{enumerate}

加入一条定理:
\begin{theorem}[定理名称:拉格朗日]
    定理内容:我想摆烂
\end{theorem}

试试建立新的环境
\begin{lemma}[theorem]{引理}
    这是一个引理
\end{lemma}
\begin{example}[theorem]{例}
    这是一个example
\end{example}
\begin{proposition}[theorem]{命题}
    这是一个命题
\end{proposition}

\section{数学公式}

数学公式:参考https://www.latexlive.com/

1.行内公式,用$..$输入
若$a>0$, $b>0$, 则$a+b>0$.

2.行间公式,用$$..$$输入
若$a>0$, $b>0$,则
$$
a+b>0.
$$

3.如果要在行内使用行间公式,则在前面加入displaystyle
设$\displaystyle\lim_{n\to\infty}x_n=x$

4.上下标用$x^7$, 下标用$y_7$,  内容多用$x^{abcd2}$, $y_{acsw5d}$.

5.分式: 5分之3, $\dfrac{3}{5}$, 
为了在分子分母行间输入较小的分式,可用$a^\frac{1}{n}$代替.

6.括号直接用$(...456+85)$输入, 如果括号内的内容高度很大,
比如出现分数的情况,可以改用$\left(1+a^\frac{1}{n}\right)^n$。
在中间需要隔开时,使用$\left(a^3\middle|x_6\right)$。
输入大括号时需要用\{大括号\}, 其中的反斜为转移字符。

7.加粗公式,注意不是文字,只针对公式: \bm{$3^4$}, \bm{$a^4$}, \textsl{\bm{$a^4$}}, \textsl{\bm{$a^x$}}

8.分段函数:
$$
f(x)=\begin{cases}
    x, & x>0, \\
    -x, & x\leq 0.
\end{cases}
$$

9.多行公式:
$$
\begin{aligned}
    a & =b+c \\
    & =d+e \\
    & = s+2
\end{aligned}
$$

10.行列式和矩阵 bmatrix和pmatrix
$$
\begin{bmatrix}
    a & b \\
    c & d  
\end{bmatrix}
$$
$$
\begin{pmatrix}
    1&2&4&8 \\
    3&a&9&a \\
    d&x^2&x&y_5
\end{pmatrix}
$$
$$
\begin{vmatrix}
    a&3 \\
    b&t^3
\end{vmatrix}
$$

\end{document}


\newpage
插入图片的使用方式:
\begin{figure}[htbp] % htbp表示自动选择插入图片的最优位置
    \centering % 设置让图片居中
    \includegraphics[width=8cm]{pics/雷电将军.jpg} % 这是图片宽度为8cm
    \caption{图片标题:雷神}
\end{figure}

\newpage
插入表格的使用方式:参考https://www.tablesgenerator.com/
\begin{table}[htbp]
    \centering 
    \caption{表格标题}
    \begin{tabular}{lllll}
    32  &     & \textbf{fsa} & sa  &     \\
    543 & fds & f            & d   & fds \\
    5   & fds & fds          &     & 4.2 \\
    4   &     & 6.065        & 2.5 &    
    \end{tabular}
\end{table}

加入列表,分为
1.无序列表
2.有序列表
3.描述description
\begin{enumerate}
    \item[(1)] 这是第一点;
    \item[(2)] 这是第二点,
    \item[(3)] 这是第三点。
\end{enumerate}

加入一条定理:
\begin{theorem}[定理名称:拉格朗日]
    定理内容:我想摆烂
\end{theorem}

试试建立新的环境
\begin{lemma}[theorem]{引理}
    这是一个引理
\end{lemma}
\begin{example}[theorem]{例}
    这是一个example
\end{example}
\begin{proposition}[theorem]{命题}
    这是一个命题
\end{proposition}

\section{数学公式}

数学公式:参考https://www.latexlive.com/

1.行内公式,用$..$输入
若$a>0$, $b>0$, 则$a+b>0$.

2.行间公式,用$$..$$输入
若$a>0$, $b>0$,则
$$
a+b>0.
$$

3.如果要在行内使用行间公式,则在前面加入displaystyle
设$\displaystyle\lim_{n\to\infty}x_n=x$

4.上下标用$x^7$, 下标用$y_7$,  内容多用$x^{abcd2}$, $y_{acsw5d}$.

5.分式: 5分之3, $\dfrac{3}{5}$, 
为了在分子分母行间输入较小的分式,可用$a^\frac{1}{n}$代替.

6.括号直接用$(...456+85)$输入, 如果括号内的内容高度很大,
比如出现分数的情况,可以改用$\left(1+a^\frac{1}{n}\right)^n$。
在中间需要隔开时,使用$\left(a^3\middle|x_6\right)$。
输入大括号时需要用\{大括号\}, 其中的反斜为转移字符。

7.加粗公式,注意不是文字,只针对公式: \bm{$3^4$}, \bm{$a^4$}, \textsl{\bm{$a^4$}}, \textsl{\bm{$a^x$}}

8.分段函数:
$$
f(x)=\begin{cases}
    x, & x>0, \\
    -x, & x\leq 0.
\end{cases}
$$

9.多行公式:
$$
\begin{aligned}
    a & =b+c \\
    & =d+e \\
    & = s+2
\end{aligned}
$$

10.行列式和矩阵 bmatrix和pmatrix
$$
\begin{bmatrix}
    a & b \\
    c & d  
\end{bmatrix}
$$
$$
\begin{pmatrix}
    1&2&4&8 \\
    3&a&9&a \\
    d&x^2&x&y_5
\end{pmatrix}
$$
$$
\begin{vmatrix}
    a&3 \\
    b&t^3
\end{vmatrix}
$$

\end{document}


\newpage
插入图片的使用方式:
\begin{figure}[htbp] % htbp表示自动选择插入图片的最优位置
    \centering % 设置让图片居中
    \includegraphics[width=8cm]{pics/雷电将军.jpg} % 这是图片宽度为8cm
    \caption{图片标题:雷神}
\end{figure}

\newpage
插入表格的使用方式:参考https://www.tablesgenerator.com/
\begin{table}[htbp]
    \centering 
    \caption{表格标题}
    \begin{tabular}{lllll}
    32  &     & \textbf{fsa} & sa  &     \\
    543 & fds & f            & d   & fds \\
    5   & fds & fds          &     & 4.2 \\
    4   &     & 6.065        & 2.5 &    
    \end{tabular}
\end{table}

加入列表,分为
1.无序列表
2.有序列表
3.描述description
\begin{enumerate}
    \item[(1)] 这是第一点;
    \item[(2)] 这是第二点,
    \item[(3)] 这是第三点。
\end{enumerate}

加入一条定理:
\begin{theorem}[定理名称:拉格朗日]
    定理内容:我想摆烂
\end{theorem}

试试建立新的环境
\begin{lemma}[theorem]{引理}
    这是一个引理
\end{lemma}
\begin{example}[theorem]{例}
    这是一个example
\end{example}
\begin{proposition}[theorem]{命题}
    这是一个命题
\end{proposition}

\section{数学公式}

数学公式:参考https://www.latexlive.com/

1.行内公式,用$..$输入
若$a>0$, $b>0$, 则$a+b>0$.

2.行间公式,用$$..$$输入
若$a>0$, $b>0$,则
$$
a+b>0.
$$

3.如果要在行内使用行间公式,则在前面加入displaystyle
设$\displaystyle\lim_{n\to\infty}x_n=x$

4.上下标用$x^7$, 下标用$y_7$,  内容多用$x^{abcd2}$, $y_{acsw5d}$.

5.分式: 5分之3, $\dfrac{3}{5}$, 
为了在分子分母行间输入较小的分式,可用$a^\frac{1}{n}$代替.

6.括号直接用$(...456+85)$输入, 如果括号内的内容高度很大,
比如出现分数的情况,可以改用$\left(1+a^\frac{1}{n}\right)^n$。
在中间需要隔开时,使用$\left(a^3\middle|x_6\right)$。
输入大括号时需要用\{大括号\}, 其中的反斜为转移字符。

7.加粗公式,注意不是文字,只针对公式: \bm{$3^4$}, \bm{$a^4$}, \textsl{\bm{$a^4$}}, \textsl{\bm{$a^x$}}

8.分段函数:
$$
f(x)=\begin{cases}
    x, & x>0, \\
    -x, & x\leq 0.
\end{cases}
$$

9.多行公式:
$$
\begin{aligned}
    a & =b+c \\
    & =d+e \\
    & = s+2
\end{aligned}
$$

10.行列式和矩阵 bmatrix和pmatrix
$$
\begin{bmatrix}
    a & b \\
    c & d  
\end{bmatrix}
$$
$$
\begin{pmatrix}
    1&2&4&8 \\
    3&a&9&a \\
    d&x^2&x&y_5
\end{pmatrix}
$$
$$
\begin{vmatrix}
    a&3 \\
    b&t^3
\end{vmatrix}
$$

\end{document}


\newpage
插入图片的使用方式:
\begin{figure}[htbp] % htbp表示自动选择插入图片的最优位置
    \centering % 设置让图片居中
    \includegraphics[width=8cm]{pics/雷电将军.jpg} % 这是图片宽度为8cm
    \caption{图片标题:雷神}
\end{figure}

\newpage
插入表格的使用方式:参考https://www.tablesgenerator.com/
\begin{table}[htbp]
    \centering 
    \caption{表格标题}
    \begin{tabular}{lllll}
    32  &     & \textbf{fsa} & sa  &     \\
    543 & fds & f            & d   & fds \\
    5   & fds & fds          &     & 4.2 \\
    4   &     & 6.065        & 2.5 &    
    \end{tabular}
\end{table}

加入列表,分为
1.无序列表
2.有序列表
3.描述description
\begin{enumerate}
    \item[(1)] 这是第一点;
    \item[(2)] 这是第二点,
    \item[(3)] 这是第三点。
\end{enumerate}

加入一条定理:
\begin{theorem}[定理名称:拉格朗日]
    定理内容:我想摆烂
\end{theorem}

试试建立新的环境
\begin{lemma}[theorem]{引理}
    这是一个引理
\end{lemma}
\begin{example}[theorem]{例}
    这是一个example
\end{example}
\begin{proposition}[theorem]{命题}
    这是一个命题
\end{proposition}

\section{数学公式}

数学公式:参考https://www.latexlive.com/

1.行内公式,用$..$输入
若$a>0$, $b>0$, 则$a+b>0$.

2.行间公式,用$$..$$输入
若$a>0$, $b>0$,则
$$
a+b>0.
$$

3.如果要在行内使用行间公式,则在前面加入displaystyle
设$\displaystyle\lim_{n\to\infty}x_n=x$

4.上下标用$x^7$, 下标用$y_7$,  内容多用$x^{abcd2}$, $y_{acsw5d}$.

5.分式: 5分之3, $\dfrac{3}{5}$, 
为了在分子分母行间输入较小的分式,可用$a^\frac{1}{n}$代替.

6.括号直接用$(...456+85)$输入, 如果括号内的内容高度很大,
比如出现分数的情况,可以改用$\left(1+a^\frac{1}{n}\right)^n$。
在中间需要隔开时,使用$\left(a^3\middle|x_6\right)$。
输入大括号时需要用\{大括号\}, 其中的反斜为转移字符。

7.加粗公式,注意不是文字,只针对公式: \bm{$3^4$}, \bm{$a^4$}, \textsl{\bm{$a^4$}}, \textsl{\bm{$a^x$}}

8.分段函数:
$$
f(x)=\begin{cases}
    x, & x>0, \\
    -x, & x\leq 0.
\end{cases}
$$

9.多行公式:
$$
\begin{aligned}
    a & =b+c \\
    & =d+e \\
    & = s+2
\end{aligned}
$$

10.行列式和矩阵 bmatrix和pmatrix
$$
\begin{bmatrix}
    a & b \\
    c & d  
\end{bmatrix}
$$
$$
\begin{pmatrix}
    1&2&4&8 \\
    3&a&9&a \\
    d&x^2&x&y_5
\end{pmatrix}
$$
$$
\begin{vmatrix}
    a&3 \\
    b&t^3
\end{vmatrix}
$$

\end{document}
